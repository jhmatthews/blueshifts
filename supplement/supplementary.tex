%%%%%%%%%%%%%%%%%%%%%%%%%%%%%%%%%%%%%%%%%%%%%%%%%%
% Basic setup. Most papers should leave these options alone.
\documentclass[fleqn,usenatbib,onecolumn]{mnras}

% MNRAS is set in Times font. If you don't have this installed (most LaTeX
% installations will be fine) or prefer the old Computer Modern fonts, comment
% out the following line
\usepackage{newtxtext,newtxmath}
% Depending on your LaTeX fonts installation, you might get better results with one of these:
% \usepackage{mathptmx}
% \usepackage{txfonts}

% Use vector fonts, so it zooms properly in on-screen viewing software
% Don't change these lines unless you know what you are doing
\usepackage[T1]{fontenc}
\usepackage{ae,aecompl}

%Fix hyperref when it runs over pages
\usepackage{etoolbox}
\makeatletter
\patchcmd\@combinedblfloats{\box\@outputbox}{\unvbox\@outputbox}{}{%
  % \errmessage{\noexpand\@combinedblfloats could not be patched}%
}%
\makeatother

%%%%% AUTHORS - PLACE YOUR OWN PACKAGES HERE %%%%%

% Only include extra packages if you really need them. Common packages are:
\usepackage{graphicx}	% Including figure files
\usepackage{amsmath}	% Advanced maths commands
\usepackage{amssymb}	% Extra maths symbols
\usepackage[usenames, dvipsnames]{color} 
\usepackage{float}
\usepackage{verbatim}
\usepackage{orcidlink}
\usepackage{xcolor}

%%%%%%%%%%%%%%%%%%%%%%%%%%%%%%%%%%%%%%%%%%%%%%%%%%

%%%%% AUTHORS - PLACE YOUR OWN COMMANDS HERE %%%%%

% Please keep new commands to a minimum, and use \newcommand not \def to avoid
% overwriting existing commands. Example:
\newcommand{\civ}{\ion{C}{iv}}
\newcommand{\civline}{\ion{C}{iv}~$\lambda$1550}
\newcommand{\sivline}{\ion{S}{iv}~$\lambda$1400}
\newcommand{\mgii}{\ion{Mg}{ii}}
\newcommand{\siv}{\ion{Si}{iv}}
\newcommand{\mgiiline}{\ion{Mg}{ii}~$\lambda$2800}
\newcommand{\heiiuv}{\ion{He}{ii}~$\lambda$1640}
\newcommand{\ciiiline}{\ion{C}{iii}]~$\lambda$1909}
\newcommand{\heii}{\ion{He}{ii}}
\newcommand{\nv}{\ion{N}{V}}
\newcommand{\nvline}{\ion{N}{V}~$\lambda$1240}
\newcommand{\oiline}{\ion{O}{i}~$\lambda$6300}
\newcommand{\oiiiline}{\ion{O}{iii}~$\lambda\lambda$4960,5008}
\newcommand{\view}{15} % fiducial viewing angle
\newcommand{\Jamesred}[1]{\textcolor{red}{JM: #1}}
\newcommand{\James}[1]{\textcolor{cyan}{JM: #1}}

%%%%%%%%%%%%%%%%%%%%%%%%%%%%%%%%%%%%%%%%%%%%%%%%%%

%%%%%%%%%%%%%%%%%%% TITLE PAGE %%%%%%%%%%%%%%%%%%%
% \begin{document}
% Title of the paper, and the short title which is used in the headers.
% Keep the title short and informative.
\title
[Supplementary material]
{
Supplementary material
}

% The list of authors, and the short list which is used in the headers.
% If you need two or more lines of authors, add an extra line using \newauthor

\author[J. H. Matthews et al.]
{James~H.~Matthews$^{\orcidlink{0000-0002-3493-7737}}$,$^{1}$\thanks{james.matthews@physics.ox.ac.uk}
Jago Strong-Wright$^{\orcidlink{0000-0002-7174-5283}}$,$^{2,3}$
Christian Knigge$^{\orcidlink{0000-0002-1116-2553}}$,$^{4}$
Paul Hewett$^{\orcidlink{0000-0002-6528-1937}}$,$^{3}$
\newauthor
Matthew J. Temple$^{\orcidlink{0000-0001-8433-550X}}$,$^{5}$ 
Knox S. Long$^{\orcidlink{0000-0002-4134-864X}}$,$^{6,7}$ 
Amy L. Rankine$^{\orcidlink{0000-0002-2091-1966}}$,$^{8}$
Matthew Stepney$^{\orcidlink{0000-0002-7711-0537}}$,$^{4}$
Manda Banerji$^{\orcidlink{0000-0002-0639-5141}}$$^{4}$ \newauthor
and
Gordon T. Richards$^{\orcidlink{0000-0002-1061-1804}}$$^{9}$
\\$^1$Department of Physics, Astrophysics, University of Oxford, Denys Wilkinson Building, Keble Road, Oxford, OX1 3RH, UK
\\$^2$DAMTP, Centre for Mathematical Sciences, University of Cambridge, Wilberforce Road, Cambridge CB3 OWA, UK
\\$^3$Institute of Astronomy, University of Cambridge, Madingley Road, Cambridge, CB3 0HA, UK
\\$^{4}$School of Physics \& Astronomy, University of Southampton, Southampton SO17 1BJ, UK
\\$^{5}$Instituto de Estudios Astrof\'{\i}sicos, Universidad Diego Portales, Av. Ej\'ercito Libertador 441, Santiago 8370191, Chile\\
$^{6}$Space Telescope Science Institute, 3700 San Martin Drive, Baltimore, MD, 21218, USA\\
$^{7}$Eureka Scientific Inc., 2542 Delmar Avenue, Suite 100, Oakland, CA, 94602-3017, USA\\
$^{8}$Institute for Astronomy, University of Edinburgh, Royal Observatory, Blackford Hill, Edinburgh EH9 3HJ, UK\\
$^{9}$Department of Physics, Drexel University, 32 S. 32nd Street, Philadelphia, PA 19104, USA\\
}

% These dates will be filled out by the publisher
\date{\today}

% Enter the current year, for the copyright statements etc.
\pubyear{2023}

% Don't change these lines
\begin{document}
\label{firstpage}
\pagerange{\pageref{firstpage}--\pageref{lastpage}}
\maketitle

% Abstract of the paper
\begin{abstract}
Supplementary material for the paper {\sl ``A disc wind model for C\textsc{iv} emission line blueshifts in quasars"} by J. H. Matthews et al., submitted to MNRAS. 
\end{abstract}


\renewcommand{\thefigure}{S\arabic{figure}}
\section{\civ\ Line profiles for all models}

In Figs.~\ref{fig:all_line_profiles1}, \ref{fig:all_line_profiles2} and \ref{fig:all_line_profiles3} we show \civline\ line profiles for the entire simulation grid, with 81 models shown for each value of $\alpha$, the acceleration exponent. As in Fig.~3 in the main text, the line profiles are colour coded by inclination and each run is labelled with a unique numeric identifier. 

\begin{figure*}
    \centering
    \includegraphics[width=\linewidth]{supplement/allprofiles_alpha0.5.png}
    \caption{All continuum-normalised \civ\ line profiles as a function of velocity and at a range of inclinations for $\alpha=0.5$. The colour scheme is different for each wind inner opening angle $\theta_{\rm min}$, with the polar, intermediate and equatorial winds shown using different colour schemes. The colour-map intensity denotes inclination, $\theta_i$.}
    \label{fig:all_line_profiles1}
\end{figure*}
\begin{figure*}
    \centering
    \includegraphics[width=\linewidth]{supplement/allprofiles_alpha1.png}
    \caption{All \civ\ line profiles as a function of inclination for $\alpha=1$.}
    \label{fig:all_line_profiles2}
\end{figure*}
\begin{figure*}
    \centering
    \includegraphics[width=\linewidth]{supplement/allprofiles_alpha1.5.png}
    \caption{All \civ\ line profiles as a function of inclination for $\alpha=1.5$.}
    \label{fig:all_line_profiles3}
\end{figure*}

% Don't change these lines
\bsp	% typesetting comment
\label{lastpage}
\end{document}